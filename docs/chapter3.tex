\section {Организация процесса сбора данных}
Здесь будут описаны принцип реализации сбора логов из образцов и построенное окружение. Поскольку
хранение и обработку логов нельзя осуществлять на той же ОС, где будут исполняться вирусные образцы,
данная деятельность будет осуществляться на специально сконфигурированной для этого ОС - "агенте". Существует два варианта организации работы схемы:
\begin {enumerate}
	\item Используя реальные ПК
	\item Используя виртуальные машины
\end {enumerate}
У варианта с реальным ПК есть свои преимущества, например, исчезает необходимость скрывать исполнение
образца в виртуальной машине, однако такой подход требует дополнительного ПК, и в случае,
если число агентов/ элементов окружения будет необходимо увеличить, нескольких. Восстановление агента
до изначального состояния может быть осуществлено с использованием специализированного ПО (FOG, fogproject.org).

В случае с виртуальными машинами, одного ПК будет достаточно. Однако, во-первых, это потребует дополнительных ресурсов самого ПК, во-вторых, существует множество техник для обнаружения исполнения кода в виртуальной среде, как специфичных для гипервизора, так и общих. Набор примерных техник обнаружения описан в  книге \cite{MALWAREANALYSIS} и может заключаться как в поиске присутствия конкретных имён процессов/ ключов реестра, так и простом сканировании памяти на наличие строк "VMWare".  Как описано в статье \cite {VMMYTHS}, создатели гипервизоров разрабатывают ПО таким образом, чтобы уменьшить нагрузку,  необходимую для запуска ОС в виртуальной среде, следовательно, различное поведение некоторых машинных инструкций в виртуальной среде и в реальной является следствием таковых упрощений реализации (например, VMWare эмулирует чипсет Intel 440BX для всех типов виртуальных машин, поскольку эмуляция всё появляющихся моделей и совместимого с ними оборудования нетривиальна). Для проверки возможных способов обнаружения виртуальной машины можно воспользоваться pafish (https://github.com/a0rtega/pafish). Однако, не следует считать наличие таких техник огромным недостатком использования виртуальных машин, т.к. в настоящее время вследствие простоты развёртывания и восстановления использование виртуальной инфраструктуры набирает популярность не только в целях исследования вирусов, но и в обычных организациях. Дальнейшее присутствие таких техник во вредоносном ПО будет наносить вред только лицам, его создающим.

Учитывая вышеупомянутые минусы виртуальных машин, всё-таки, восстановление их до чистого состояния путём использования снапшотов (гипервизор хранит на жёстком диске реального ПК разницу виртуальной оперативной памяти и разницу в секторах виртуального жёсткого диска в виде файлов, и просто восстанавливает файлы до исходного состояния)  представляется слишком удобной. В качестве виртуальной среды была выбрана Oracle VirtualBox, поддерживающая снапшоты в бесплатной версии по сравнению с аналогичной продукцией от VmWare. Также, используется API VirtualBox для передачи файлов через предоставляемый Oracle модуль-обёртку на Python vboxapi (однако, факт использования данной реализации может быть использован в качестве одной из вышеупомянутых специфичных для VirtualBox техник обнаружения виртуальной машины, т.к. требует установки служб VirtualBox guest additions на агента, и поэтому возможно изменение подхода в дальнейшем).

По аналогии с последовательностью, описанной в статье \cite {MASSMALWARE}, получена следующая последовательность шагов, представляющая собой цикл исполнения образцов в VirtualBox.

\begin {itemize}
	\item Предварительная конфигурация агента
	\item Загрузка образца
	\item Запуск отслеживающих инструментов и исполнение образца
	\item Получение логов
\end {itemize}

\subsection {Предварительная конфигурация агента}
Здесь большинство действий было выполнено однократно, после чего был создан снапшот состояния с настроенным окружением, готового к запуску. Установлен отладчик Immunity Debugger, изменены его настройки, установлен Python 2.7.1, необходимый для работы отладчика, ApateDNS (утилита, позволяющая перенаправлять все DNS запросы на указанный IP адрес), .Net Framework 4.0, 4.5 (некоторые образцы построены с его использованием и не работают без него), выключен файрвол Windows и служба контроля учётных записей (User account control).

Далее на этом шаге просто происходит возврат в "чистое состояние"

\subsection {Загрузка образца}
Используя vboxapi, реализован скрипт, скачивающий с основного ПК образец и скрипты, управляющие сбором логов.
\subsection {Запуск отслеживающих инструментов и исполнение образца}
Поскольку снапшот восстанавливаются из готового состояния, ApateDNS уже запущена и настроена на сервер INetSim.
\paragraph {PyCommands}
Immunity Debugger поддерживает реализацию определённой последовательности действий, описанную в скрипте на ЯП Python - PyCommand, которая может быть вызвана через его консоль. Одной из причин, послужившей для выбора его в качестве отладчика послужило наличие уже готовой реализации техник обхода защиты от отладки в PyCommand hidedebug, функционал которой был встроен в используемую PyCommand'у. Также PyCommand поддерживают хуки - описанные функции, которые будут вызваны в случае возникновения настроенного на них события.

Образец исполняется в отладчике, после чего специальный скрипт собирает информацию о присутствующих в таблице импортов исполняемого файла функций и устанавливает на них точки останова. Тот же скрипт регистрирует хук, в котором описаны необходимые для получения параметры в случае вызова таковых функций.
В случае вызова функции, отладчик логирует всё в файл. Образец исполняется в течение 2 мин.
\subsection {Получение логов}
По истечении указанного промежутка времени скрипт через vboxapi забирает собранные логи о деятельности образца и выключает виртуальные машины.

 
