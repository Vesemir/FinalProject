\section* {Введение}
\addcontentsline{toc}{section}{Введение}
Проблемы информационной безопасности в настоящее время охватывают большое число специализированных областей. Каждая из них состоит из множества проблем, часть которых до сих пор не получила исчерпывающего решения. К одной из таких проблем относится проблема обнаружения вирусов и вредоносного программного обеспечения (ВПО).

Традиционные сигнатурные методы имеют высокую эффективность обнаружения, но только в том случае, если компьютерный вирус был предварительно изучен и для него была определена сигнатура\cite {GOSTVIRUS}. Суть процесса анализа файла на вредоносность при этом заключается в простом сканировании на наличие определённых уникальных последовательностей байтов. Алгоритмы работы каких-либо более сложных методов не публикуются производителями антивирусных средств в силу защиты интеллектуальной собственности и  поддержания конкурентоспособности. В связи с этим разумной выглядит идея создания, развития и поддержки программного модуля с открытым исходным кодом, который мог бы решать задачи обнаружения заранее не подвергавшихся изучению вирусов или ВПО, имеющего сходное поведение. Поскольку исходный код модуля является открытым, удобным выглядит использование при разработке системы контроля версий Git и общедоступного сайта для размещения исходных кодов программных средств github.com.

При этом подход к созданию модуля основывается на двух принципах:
\begin {enumerate}
	\item Добавление знаний о новом вирусе X не должно заключаться лишь в добавлении возможности обнаружения копий X (как это происходит при использовании сигнатур). Критерий сравнения цепочек должен позволять предлагать в качестве опасной последовательности вирус Y, отличающийся от цепочки вызовов вируса X добавлением некоторых дополнительных вызовов, а также вирус Z, отличающийся от цепочки вызовов вируса X удалением некоторых вызовов.
	\item Также это должно позволять обнаруживать отдельные группы цепочек вызовов, схожие в X и вирусах других типов. Например, в случае модульной организации построения вредоносного ПО переиспользование отдельных модулей может дать возможность обнаружения новых образцов. Если посмотреть детальнее, такой подход потенциально способен обнаруживать именно конкретные техники в ВПО, наличие которых может говорить о подозрительности поведения, при этом сами техники потенциально должны поддаваться интерпретации. У многих профессиональных программистов есть выработанный почерк написания кода, и это тоже даёт возможность поиска отдельных техник, однако, такая точность находится за рамками данной работы.
\end {enumerate}
На настоящий момент наиболее распространённой ОС, используемой для управления рабочими ПК, всё ещё является Windows. В связи с этим к рассматриваемому в настоящей работе вредоносному ПО также будут относиться только образцы, скомпилированные для работы под платформой Windows.

В исследовательском разделе будут подробнее рассмотрены причины, не позволяющие воспользоваться лишь статическими методами анализа для решения данной задачи, будут предложены варианты, позволяющие найти решение, а также будут рассмотрены примеры некоторых техник (цепочек вызовов), используемых в вирусах, и вызовов, имеющих большую значимость, чем остальные.

В специальном разделе будет подробно описана разработанная схема получения логов, описывающая способ получения цепочек, указан выбранный формат для компактного хранения данных, применено несколько базовых техник для борьбы с обнаружением вирусом исследующей его среды, описан выбранный критерий схожести и алгоритм сравнения цепочек, а также описаны принятые решения некоторых проблем с производительностью алгоритма при больших размерах входных данных.

Вопросы безопасности жизнедеятельности имеют большое значение во всех производственных сферах, в том числе и на этапах разработки, внедрения и эксплуатации программного обеспечения.

В разделе <<Безопасность жизнедеятельности>> будет проанализирован синдром компьютерного состояния пользователя, разработаны мероприятия по снижению компьютерного состояния пользователя, а также проведена экологическая оценка и рассмотрена технологическая схема переработки компьютерного лома.

В экономическом разделе будет произведено планирование разработки программного продукта с построением графика, произведён расчёт сметы затрат на разработку программного продукта, а также произведён расчёт эффективности использования программного продукта.

В технологическом разделе будет проверена работоспособность разработанного модуля на полученных из открытых источников образцах и сделаны выводы по результатам прогона.