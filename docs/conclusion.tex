\section* {ЗАКЛЮЧЕНИЕ}
\addcontentsline{toc}{section}{ЗАКЛЮЧЕНИЕ}
В рамках исследовательского раздела был выбран подход, который в дальнейшем в рамках специального раздела позволил разработать программный модуль, способный реализовывать поставленные для него задачи --- обнаружение потенциально вредоносных цепочек вызовов от исследуемых образцов, до определённой степени похожих на заранее собранные цепочки, описывающие поведение вирусов. Также, одним из основных преимуществ выбранного подхода является постепенное накопление знаний о поведении вредоносного ПО при постепенном увеличении базы знаний, состоящей из цепочек, что в итоге должно позволить увеличить точность обнаружения при увеличении срока эксплуатации программного модуля. Выбранный формат хранения данных в hdf5 достаточно компактен и позволяет при желании производить его параллельную обработку в многоядерных или многопроцессорных системах, в случае, если размер базы знаний достигнет многих миллионов вредоносных образцов. Открытость исходного кода модуля даёт возможность добавить использование для сбора логов каких-либо других сред виртуализации (Citrix Xen, VMware VSphere) в случае, если будет необходимо более эффективное использование аппаратных ресурсов хоста, на котором выполняется данный сбор.

В разделе «Безопасность жизнедеятельности» был проанализирован синдром компьютерного состояния пользователя, разработаны мероприятия по снижению компьютерного состояния пользователя, а также рассмотрена технологическая схема переработки компьютерного лома и произведена её экологическая оценка.

В организационно-экономическом разделе определены стадии разработки ПС, состав работ, рассчитано время, требующееся на проведение исследования и тестирование, построен ленточный график разработки ПО, определены затраты на разработку ПО, приведены основные технико-экономические показатели проведения исследования. 
Трудоемкость разработки, согласно расчетам, составит 104 человеко-дней, продлится 134 календарных дня, а затраты на нее составят 463843.6 рублей.

В рамках технологического раздела была произведена пробная эксплуатация разработанного программного модуля, что позволило проверить эффективность выбранного подхода на реальных вредоносных образцах. Однако, к полученному довольно высокому проценту обнаружения вредоносных цепочек следует относиться с осторожностью, так как параметры программы не были откалиброваны на ПО, не имеющем вредоносного поведения, и с этой точки зрения возможно проведение дополнительных работ с целью уменьшения ложных положительных срабатываний на цепочках вызовов, которые являются распространёнными в ПО любого типа и поведения.

В итоге, предложенный подход к решению поставленной проблемы и его реализация дают возможность проведения анализа исполняемых файлов ОС Windows под платформу x86 (ограничение платформы связано только с применяемым отладчиком, и может быть снято в случае использования, например x64dbg, поддерживающего платформу x64; однако, на момент написания работы создателями отладчика всё ещё не был реализован интерфейс управления через используемый в качестве основного языка в данном проекте ЯП Python) с целью выявления вредоносного ПО и определения его возможного поведения на СВТ, если таковой вредоносный образец был выявлен. Использование модуля должно производиться совместно с сертифицированными антивирусными средствами защиты, что в итоге потенциально позволяет увеличить процент выявления вирусов в случае, если эвристические алгоритмы обнаружения в используемом антивирусе кардинально отличаются от реализованных в настоящем модуле. Наконец, настоящий модуль можно рассматривать, как дополнительный слой <<защиты в глубину>>, дающий дополнительные возможности для решения задачи обнаружения и исследования неизвестного ВПО в автоматическом режиме, причём потребность в таких средствах будет расти с каждым годом в связи с постоянно увеличивающейся скоростью выпуска новых вирусов.