\section {Алгоритм сравнения цепочек}
 бла-блу
\subsection {Приведение данных к форме, удобной для обработки алгоритмом}
Собранные текстовые логи работы вредоносных образцов являются достаточно информативными с точки зрения ручного анализа, однако для автоматической обработки их следует привести к более компактной и быстрее обрабатываемой форме (как правило, в любых языках программирования операции над строками медленнее операций над числовыми данными). Для анализа цепочек было решено хранить информацию о вызовах как массив из int32 (4 байта на один вызов). Такая форма достаточна для хранения более чем 4 миллионов различных вызовов, а также она поддерживается пакетом numpy ЯП Python. При этом словарь, отображающий фактические имена (и, возможно, параметры) функций в числа хранится отдельно. Это позволяет в дальнейшем возвращать более подробную информацию о найденных совпадениях цепочек. Реализация данного преобразования хранится в lib/pander.py.
\subsection {Алгоритм Смита-Ватермана}
Для сравнения как бинарных, так и текстовых строк (или в нашем случае --- последовательности чисел) классической метрикой является расстояние Левенштейна (также называемое дистанцией редактирования). Между двумя строками дистанцией редактирования будет являться число базовых операций, применяемых к одной строке, чтобы получить другую. К базовым операциям относятся:
\begin {itemize}
	\item Вставка одного символа
	\item Удаление одного символа
	\item Замена одного символа на другой
\end {itemize}
Эта  метрика используется в алгоритме Смита-Ватермана, впервые опубликованном в \cite {LOCALALIGNMENT}. Несмотра на то, что придуман он был для нахождения участков похожих цепочек нуклеотидных последовательностей и относится к генетике, некоторые авторы \cite{BLACKBOOK} относятся к проблеме компьютерных вирусов как к искусственной форме жизни (если точнее, то self-replicating automata, "самовоспроизводящийся автоматический механизм"), поэтому идеи ближе, чем может показаться на первый взгляд.

Существует два основных подхода к сравнению последовательностей: глобальное и локальное выравнивание. Применение глобального выравнивания требует примерно одинаковой длины цепочек, что нельзя гарантировать в случае исследования произвольных образцов. Использование глобального выравнивания можно было бы отнести к поиску практически идентичных копий вредоносных образцов. Алгоритм Смита-Ватермана применяет локальное выравнивание, что ближе к идее нахождения каких-либо распространённых сходных отрезков последовательностей вызовов функций API, потенциально обладающих вредоносным воздействием. Он берёт две последовательности произвольной длины и находит оптимальное выравнивание в любом месте последовательности согласно задаваемой матрице замен, определяющей изменение счёта текущего выравнивания. 

Набор шагов алгоритма заключается в следующем:
\begin {enumerate}
	\item Вычисляем матрицу выравниваний
	
	Двумерная матрица размерностью в ( длина первой последовательности X длина второй последовательности) инициализируется нулями. 
\end {enumerate}

